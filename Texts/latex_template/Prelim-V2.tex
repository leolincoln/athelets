\begin{comment}
Commented code

We now present an overview of some of the techniques for obtaining location data, which we assume and rely upon in this work. Specifically,
we discuss the main features of road-side sensors and GPS devices.
Subsequently, we proceed with introducing the basic terminology and notation used
in the rest of the paper.

\subsection{Road-side Sensors}

Starting in the 1920s, when the traffic signals were still
manually controlled, several generations of sensor types have been
developed and deployed along road segments in various states -- all for the purpose of more efficient 
traffic management. The types of such sensors vary from the older
pressure-sensitive ones introduced in 1931, to more modern laser-based sensors
sensors~\cite{trafficmanual} and quite a few 
%%%Contrary to the GPS-based data acquisition techniques where each
%%%data source is isolated, the roadside sensors are usually
%%%connected hierarchically to a server and send their sampled data to traffic
%%%control center~\cite{roadside-server}. Compared with GPS system,
%%%the roadside sensors have better measurement accuracy, higher
%%%sampling frequency and shorter response time, which enables their
%%%use in real time traffic information analysis and control.
different types have been commercialized and
used in day-to-day practical settings. For example, the AMR sensor~\cite{honeywell}
developed by Honeywell is a type of magnetic sensor with low cost.
The WiEye~\cite{easysense} is a passive infrared sensor that can be
installed on top of motes to sense road condition. The variation of
sensing technologies may affect the manner of how a motion is modeled, in order to
capitalize on the capabilities of a particular type of sensor. In this paper, the
data model for roadside sensor that we adopt is based on TruSense T-Series,
manufactured by Laser Technology Inc.~\cite{lasertech} -- a kind of
active infrared sensor with a very accuracy as well as a high sampling
rate.

\begin{table*}[ht]
\centering \caption{Comparison among different types of sensor}
\begin{tabular}{|c|c|c|c|c|c|c|} \hline
Sensor technology&Count&Presence&Speed&Output
Data&Classification&Multiple Lane detection\\ \hline
\texttt{Inductive loop} & \cmark & \cmark & \cmark & \cmark &
\cmark & \xmark \\ \hline \texttt{Magnetometer (two axis fluxgate)}
& \cmark & \cmark & \cmark & \cmark & \xmark & \xmark \\ \hline
\texttt{Magnetic induction coil} & \cmark & \cmark & \cmark &
\cmark & \xmark & \xmark\\ \hline \texttt{Microwave radar}
& \cmark & \cmark & \cmark & \cmark & \cmark & \cmark
\\ \hline \texttt{Active infrared} & \cmark & \cmark & \cmark &
\cmark & \cmark & \cmark \\ \hline \texttt{Passive infrared}
& \cmark & \cmark & \cmark & \cmark & \xmark & \xmark \\ \hline
\texttt{Ultrasonic} & \cmark & \cmark & \xmark & \cmark & \xmark &
\xmark \\ \hline \texttt{Acoustic array} & \cmark & \cmark & \cmark
& \cmark & \xmark &\cmark \\ \hline \texttt{Video image processor}
& \cmark & \cmark & \cmark & \cmark & \cmark & \cmark \\ \hline
\end{tabular}
\label{tab1}
\end{table*}


Table~\ref{tab1} provides a summary of features of several different
types of roadside sensors~\cite{trafficmanual}. As can be seen, 
all of the popular and commercially available types can detect the presence and speed of vehicles, as well as provide a
count value for the number of vehicles that have been detected in their
sensing range. However, very few types provide more detailed sensing
capabilities, such as classification and multiple lanes detection. We note that, unlike the 
GPS-based data, the location-in-time information obtained from the roadside sensors has not been
exploited extensively in MOD context.
%%%data obtained with roadside sensors has not been  In this work,
%%%we focus on detection of a presence of a moving object in the sensing range.



\subsection{GPS-based Spatio-Temporal Uncertainty}

When location is obtained via an on-board GPS device, plethora of works 
from the MOD literature have addressed efficient processing of various categories of
continuous spatio-temporal queries. Specifically, the problem of incorporating the uncertainty of the 
location has been treated from a couple of basic stand-points.  

One approach for modeling spatio-temporal
uncertainty of moving objects is the, so called, sheared cylinder
model. The main assumption is that at any time
instant $t_i$, the object's location is inside a given disk with a
fixed radius, centered at the expected location at $t_i$. For time
values different from sampling ones, the expected location is obtained via
linear interpolation~\cite{MyTODS04}. However, this model is geared towards
past/historic trajectories.


\begin{figure}
    \centering
    \includegraphics[width=2.8in]{figure/bead1.png}
    \caption{Bead and Ellipse Model}
    \label{Beads-fig}
\end{figure}

%%The time-geography~\cite{hagerstrand} ideas from the 1970s (and
%%more recently probabilistic time geography~\cite{WinterY10}) have
%%also permeated MOD research.
The implications of the fact that the
object's motion was bound by some $v_{max}$ \emph{in-between} two
updates was analyzed in~\cite{pj99}. Based on the definition as a
geometric set of 2D points, it was demonstrated that the
possible whereabouts are bound by an ellipse, with foci at the respective
point-locations of the consecutive samples.
Subsequently,
\cite{HornsE02}, presented a spatio-temporal version of the model,
naming the volume in-between two update points a {\it
bead}, and the entire uncertain trajectory, a {\it necklace}. This model 
was actually introduced as a {\it space-time prism} in the geography literature~\cite{hagerstrand}. 
However, the first work to present a formal analysis of the
properties of the bead are~\cite{Kuijpers10}. An illustration is
provided in Figure~\ref{Beads-fig}.
Letting $d = \sqrt{(x_2 -
x_1)^2 + (y_2 - y_1)^2}$ denote the distance between the starting
location (at $t_1$) and ending location (at $t_2$), the equation
of the projected ellipse (cf. ~\cite{pj99}) is:

$$\displaylines{\frac{(2x - x_1 - x_2)^2}{v_{max}^2 (t_2 - t_1)^2}
+ \hfill{}\cr \frac{(2y - y_1 - y_2)^2}{v_{max}^2 (t_2 - t_1)^2 -
(x_2 - x_1)^2 - (y_2 - y_1)^2} = 1 }$$

The corresponding bead (or, space-time prism) is specified with the following
constraints:

\begin{equation} \label{Eq1}
\begin{cases}
t_i \leq t \leq t_{i+1} \\
(x - x_i)^2 + (y - y_i)^2 \leq [(t - t_i)v_{max}^i]^2 \\
(x - x_{i+1})^2 + (y - y_{i+1})^2 \leq [(t_{i+1} - t)v_{max}^i]^2
\end{cases}
\end{equation}

\noindent where $v_{max}$ is the maximal speed that the object can
take between $t_i$ and $t_{i+1}$. We note that, what is commonly
called {\it expected} speed in the case of crisp trajectories, now
becomes {\it minimal} expected speed in-between the updates/samples.
As shown
in Figure~\ref{Beads-fig}, at any time instant $t$ between two consecutive samples,
the possible locations of the objects are bound by the lens -- i.e., intersection of two circles
centered at the respective foci and with respective radii $v_{max}(t - t_1)$ and $v_{max}(t_2 - t)$.

If the objects are constrained to move along a road network, then
the space-time prisms are restricted in their size. Specifically,
if the segments of the road network are assumed to be edges in a
graph, then the prisms become restricted to planar
figures~\cite{EmrichKMRZ12}.

\subsection{Trajectories and Road Networks}

Throughout this paper, we consider the following definition of a trajectory:

\begin{definition} \label{Def-Trajectory}
A {\it trajectory} $Tr_i$ of a moving object with a unique identifier ({\it oID}) \enquote{\it i},
is a sequence of triplets $Tr=$ $[(L_1, t_1), (L_2, t_2),
v_{max1}] \ldots$, $[(L_{n-1}, t_{n-1}), (L_n, t_n),
v_{max\_(n-1)}]$ where each $L_i = (x_i, y_i)$ is a point in 2D space in a corresponding
reference coordinate system, and $t_i$ denotes the time instant at which the object was
at location $L_i$. When it comes to the time-values,
$i < j$ implies $t_i < t_j$, and $v_{max\_i}$ denotes the
maximum speed of the object between samples at $t_i$ and $t_{i+1}$
\end{definition}

\begin{figure}[hb]
    \centering
    \includegraphics[width=2.6in]{figure/Fig-RoadNet-Sensors.png}
    \caption{Road Segments and Sensors}
    \label{Sensor-fig}
\end{figure}


We define a road network as an {\it augmented graph} $G = (P, E_{RS})$ where $P
= \{ p_1, p_2, \ldots, p_n \}$ denotes a set of points (commonly
corresponding to intersections) and $E_{RS} = \{ r_{S1}, ..., r_{Sk} \}$ is a collection of triplets of the form
$r_{Si} = (e_i, w_{ei},  v_{ei})$ where:

\begin{itemize}

\item $e_i = (p_{i1}, p_{i2}) $ ($ \in P$ X $P$) is a "regular edge" (i.e., a link between two connected vertices)

\item $w_{ei}$ denotes the width of the road segment associated with the edge $e_i$.

\item $v_{ei}$ denotes the maximum speed associated with $r_{Si}$.

\end{itemize}

We assume that the maximum speed
in-between two consecutive location samples along a particular road segment corresponds to the
speed-limit of that segment. Geometrically speaking, the collection of all the $r_{Si}$'s is
the boundary of the Minkowski sum of each "regular edge" $e_i$ and a disk with diameter $w_{ei}$.

We also assume the existence of a collection of sensors $S = \{
s_1, s_2, ..., s_m \}$, where each sensor $s_j$ is located at a
point along the outer boundary of some road segment $r_{Si}$. Each
$s_j$ detects when (i.e., the time instant at which) a moving object crosses the line segment going
through its location and perpendicular to $e_{i}$. The concepts are
illustrated in Figure~\ref{Sensor-fig}.
\end{comment}
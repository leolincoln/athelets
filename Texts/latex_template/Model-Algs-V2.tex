We now discuss the details of the new uncertainty model resulting from combining
the GPS-based location data and the location data generated by road-side sensors.
%%%Some properties of our fused model will be discussed later.
%%%We follow with a discussion of the semantics of the basic
%%%whereabouts queries, lane-crossing and range queries.

%%%\subsection{Combined Model}

The main observation is that the road-side sensors provide additional constraints on the possible whereabouts in-between
two consecutive GPS-based samples (and vice-versa). More specifically, recall that the "traditional" bead (i.e., space-time prism)
 was defined by the system of inequalities (1) (cf. Section 2). In addition to those inequalities, we now
 have the constraint that at a particular time instant $t_{si}$, the
possible locations of a particular moving object detected by the roadside sensor are also known to be along a
given line-segment determined by:
\begin{enumerate}

\item the location of the
corresponding road-side sensor, and

\item the direction which is perpendicular to the (boundaries of the)
road segment.

\end{enumerate}

\begin{comment}
This can be formalized as:


%%%\begin{figure}[ht]
%%%   \centering
%%%    \includegraphics[width=2.6in]{figure/Fig-fused1.png}
%%%   \caption{Fusing GPS and Roadside Sensor Data}
%%%    \label{fusing1}
%%%\end{figure}

\begin{figure*}
    \centering
    \begin{subfigure}[b]{0.39\textwidth}
                \includegraphics[width=2.7in]{figure/Fig-fused1.png}
                \caption{GPS + Roadside Sensors}
                \label{fusing1-1}
    \end{subfigure}%
    \begin{subfigure}[b]{0.39\textwidth}
                \includegraphics[width=2.7in]{figure/Fig-fused2.png}
                \caption{Determining Boundaries}
                \label{fusing1-2}
    \end{subfigure}%
    \caption{Fusing GPS and Roadside Sensors Data}
    \label{fusing1}
\end{figure*}





\begin{align}
\begin{cases}
t_{i} \leqslant t \leqslant t_{i+1},\\
(x - x_{i})^{2} + (y - y_{i})^{2} \leqslant (t - t_{i})^{2}v^{2}_{max},\\
(x - x_{i+1})^{2} + (y - y_{i+1})^{2} \leqslant (t_{i+1} - t)^{2}v^{2}_{max},\\
y = m_{i}x + b_{i}, \text{when }t = t_{si}\\
t_{i} \leqslant t_{si} \leqslant t_{i+1}.
\end{cases}
\end{align}

The system of constraints (2) is illustrated in Figure~\ref{fusing1}:
Specifically, as shown in Figure~\ref{fusing1-1},
the original GPS-based locations $L_1$ and $L_2$ would yield a 2D projection which is
an ellipse having them as foci (light-grey shaded shape in Figure~\ref{fusing1-1}) -- denote it $El_1$.
Due to the road-side sensor, the possible locations of
the moving object at $t_{s1}$ can only be along the {\it portion} of line segment originating in $(x_{s1}, y_{s1})$,
perpendicular to the boundaries of the road segment, {\it and intersecting $El_1$} -- i.e., along the portion of the
line segment $\overline{L_{1}'L_{1}''}$.
Clearly, that intersection has an uncountably many points, and we show
3 such points in Figre~\ref{fusing1-1} -- $L_{11}$, $L_{12}$ and $L_{13}$.
Each such point, in turn, can be used as a "generator" for two
more space-time prisms: one originating in $L_1$, and the other terminating at $L_2$.
The corresponding 2D projections (ellipses) are shown in Figure~\ref{fusing1-1} for $L_{11}$, $L_{12}$ and $L_{13}$.
The most important implication is that when combining the original ellipse
$El_1$ with the uncountably infinite collection of the ellipses with
one of the foci along the line segment due to the road-side sensors,
the additional constraint induces a significant amount of a ``dead-space" in $El_1$.
A more detailed illustration of the valid range for selecting the points that will generate the
infinite collection of (pairs of) new beads is given in Figure~\ref{fusing1-2}. Recall that
at any $t_{s1}$ between the sampling times $t_1$ and $t_2$,
the object can be located inside of the lens obtained as the intersection of the circles with
radii $v_{max} (t_{s1} - t_1)$ and $v_{max} (t_2 - t_{s1})$. Hence, although the ray emanating from the
roadside sensor $s_1$ would intersect the "global boundary" (i.e., the ellipse which is the projection of the
bead) at $L_{1}'$ and $L_{1}''$, the only valid points to be considered as possible whereabouts are the ones
along (and inside) the lens. As shown in Figure~\ref{fusing1-2}, those are the points along the line segment
bounded by $L_{11}$ and $L_{13}$.

We note that there is a complementary context of having a single uncertainty source -- i.e., in contrast to having GPS-based points only.
Namely, if there were only the roadside sensors available,
then in between two detections by consecutive sensors (say, $s_1$ and $s_2$ from Figure~\ref{Sensor-fig}), the whereabouts of
a given object is bounded by the infinite union $\cup(El_{si,sj})$ of uncountably many ellipses
for which:

\begin{enumerate}
\item The first focus is some point $L_{s1}$ located on the line-segment originating at the location of $s_1$.

\item The second focus is some point $L_{s2}$ located on the line-segment originating at the location of $s_2$;

\item The distance between $L_{s1}$ and $L_{s2}$ is smaller than $v_{max} (t_{s2} - t_{s1})$ (i.e., the object could travel the
distance within the time-interval $[t_{s1}, t_{s2}]$ for the given speed limit).

\end{enumerate}


\begin{figure}[hb]
    \centering
    \includegraphics[width=2.8in]{figure/multiple.png}
    \caption{Multiple Roadside Sensors Intersecting a Bead}
    \label{fig:multiple}
\end{figure}

Incorporating the GPS-based bead in this context would either amount to the case where it intersects one (or more) of the line segments
originating at the respective sensors locations, or it has no intersection with any of them.  In the
latter case, we have a scenario in which GPS sampling frequency is higher than the sampling frequency obtained by the roadside
sensors. For such settings, the possible whereabouts will be reduced to the intersection of the $\cup(El_{si,sj})$
and the bead obtained from the GPS-based samples. In the former case, the model is a generalization of the
one corresponding to the scenario illustrated in Figure~\ref{fusing1} -- in the sense that it may be possible to have intersections of the GPS-based bead with $>1$ sensor lines, as illustrated in Figure~\ref{fig:multiple}. In the rest of this paper, we focus on detailed discussion of the scenarios in which a bead is intersected by a line segment emanating from a single roadside sensor.


We call the spatio-temporal structure induced by combining the two uncertainty sources -- GPS and roadside sensors -- a {\it Fused Bead} (FB), and it is a sixtuple $FB$ $($ $(x_{i}, y_{i}, t_i), $ $ (x_{i+1}, y_{i+1}, t_{i+1}), $ $v_{max}, t_{s}, m, b)$
consisting of:

\begin{itemize}

\item The 2 GPS-based location-in-time samples $(x_{i}, y_{i}, t_i), $ and $ (x_{i+1}, y_{i+1}, t_{i+1})$ along with the $v_{max}$ speed bound.

\item The time instant of detection of the road-side sensor.

\item  The parameters of the equation $y = mx + b$ (in a given referent coordinate system) of the line specifying the corresponding line-segment emanating from the roadside sensor and specifying the locations of the possible new foci.

\end{itemize}





\begin{figure}
    \centering
    \includegraphics[width=2.6in]{figure/fusing_mat.png}
    \caption{Outer Boundary of the Fused Uncertain Locations}
    \label{fusing:MAT}
\end{figure}

When it comes to bounding the possible whereabouts, an intuition may cause one expect that
some of the points along the intersection of the line segment with the ellipse $El_1$ may
yield possible focal points that would generate ellipses which are not fully contained inside $El_1$.
However, the set of constraints in (2) will eliminate
every portion which is outside the intersection of the original $El_1$.

We now proceed with a formal analysis of an important property of the FB model, towards which we first
recall some of the properties of the bead model presented in~\cite{Kuijpers10}. Let $B(x_i, y_i, t_i, x_{i+1}, y_{i+1}, t_{i+1}, v_{max})$
denote\footnote{The original notation in~\cite{Kuijpers10} was $(B(t_i, x_i, y_i, t_{i+1}, x_{i+1}, y_{i+1}, v_{max}))$ and we slightly modified it for consistency with the rest of the notation in this article.} the bead between two location-samples $(x_i, y_i)$ and $(x_{i+1}, y_{i+1})$ at respective times $t_i$ and $t_{i+1}$, during which the speed is bounded by $v_{max}$
%%%Hence, the FB model
%%%has the property (lemma~\ref{bead_bond}) that the original space-time prism
%%%obtained from the GPS samples, is an outer-boundary of the corresponding FB -- as illustrated
%%%in Figure~\ref{fusing:MAT}.

\begin{property}
Given $(x_i, y_i, t_i),$ and $(x_{i+1}, y_{i+1}, t_{i+1})$ with $t_i < t_{i+1}$ and
$v_{max} > 0$, any trajectory from $(x_i, y_i, t_i)$ to $(x_{i+1}, y_{i+1}, t_{i+1})$
for which the speed at any moment $t_{i} \leq t \leq t_{i+1}$ is less than $v_{max}$
is located within the bead $B(x_i, y_i, t_i, x_{i+1}, y_{i+1}, t_{i+1}, v_{max})$ and the
projection of such a trajectory on the $(x,y)$-plane is located within
$\pi_{x, y}(B(x_i, y_i, t_e, x_{i+1}, y_{i+1}, t_{i+1}, v_{max}))$. Furthermore, for any point $(x, y, t)$ in $B(x_i, y_i, t_i, x_{i+1}, y_{i+1}, t_{i+1}, v_{max})$, there exists a trajectory from $(x_i, y_i, t_i)$ to $(x_{i+1}, y_{i+1}, t_{i+1})$ which passes through $(t, x, y)$.
\label{tra-bond}
\end{property}


Property 1 explains the bounding relationship between trajectory and bead. Taking the constrain 1
into consideration, one can deduce to the following corollary:

\begin{corollary}
Any trajectory from $x_i, y_i, t_i$ to $(x_{i+1}, y_{i+1}, t_{i+1})$ which passes through a point that lies
on the boundary of the ellipse $$\displaylines{\frac{(2x - x_1 - x_2)^2}{v_{max}^2 (t_2 - t_1)^2}
+ \hfill{}\cr \frac{(2y - y_1 - y_2)^2}{v_{max}^2 (t_2 - t_1)^2 -
(x_2 - x_1)^2 - (y_2 - y_1)^2} = 1 }$$ is the longest possible trajectory.
\end{corollary}


In a similar spirit, and based on these properties of the bead model, we now have the following property regarding the FB model:

\begin{lemma}
	Any bead generated by: (1) a focal point located in the GPS-based sample, and (2) a point from the line segment
	$P_{1}P_{2}$ representing possible locations obtained via a roadside sensor, is contained within the original bead.
	\label{bead_bond}
\end{lemma}
\begin{proof}
	\begin{figure}
    \centering
    \includegraphics[width=2.8in]{figure/Proof.png} %past is 2.6
    \caption{Proof of Fused Bead Containment}
    \label{proof}
	\end{figure}
	

	We prove Lemma~\ref{bead-bond} by contradiction. Assume that $P_n$ is a point on the
	line segment $P_{1}P_{2}$ and consider the ellipse $El_2$ with foci $P_n$ and $L_1$.
	Let $A_1$ denote a point which
	lies within $El_2$ but outside the original bead $El_1$, defined by the original bead (i.e., foci $L_1$ and $L_2$, and
$v_{max}$ bounding speed). Connect the two line segments
	$L_{1}A_{1}$ and $A_{1}P_{n}$. They intersect $El_1$ at some points, denote them $A_2$ and $A_3$.
	According to Corollary 1, the polyline with two segments $L_{1}A_{3}L_{2}$ is the longest trajectory
	that the object could possibly move along from $L_1$ to $L_2$.
	However, by assumption, $A_1$ is bounded to be within
	$El_1$ which, in turn, implies that $L_{1}A_{1}P_{n}$ is a route of a valid trajectory from $L_1$ to $P_n$  and, moreover,
	$L_{1}A_{1}P_{n}L_{2}$ is a route of a valid trajectory from $L_1$ to $L_2$. However,
	since, based on the triangular inequality, $\overline{A_{3}A_{1}} + \overline{A_{1}P_{n}} > \overline{A_{3}P_{n}}$ and $\overline{A_{3}P_{n}} + \overline{P_{n}L_{2}} > \overline{A_{3}L_{2}}$,
	we have $\overline{A_{3}A_{1}}+ \overline{A{1}P_{n}} + \overline{P_{n}L_{2}} > \overline{A_{3}L_{2}}$. Based on the last inequality,
	we can conclude that $\overline{L_{1}A_{3}} + \overline{A_{3}A_{1}} + \overline{A_{1}P_{n}} + \overline{P_{n}L_{2}} > \overline{L_{1}A_{3}} + \overline{A_{3}L_{2}}$,
	which implies that the trajectory $\overline{L_{1}A_{1}P_{n}L_{2}}$ is longer than
	trajectory $\overline{L_{1}A_{3}L_{2}}$. This, however, is a contradiction to the Corollary 1 which states that
	no other valid trajectory is longer than $L_{1}A_{3}L_{2}$, and we could conclude that
	assumption on the existence of point $A_1$ is not valid.
\end{proof}

Lemma 1 demonstrates that whenever there is a location sampling from a roadside sensor in-between two GPS-based location samples, the possible locations by the FB  model are contained within the set of possible locations bounded by original GPS-based bead. The main implication of Lemma 1 is in the conclusion that the FB will not introduce any false positives -- in comparison with the traditional bead -- when determining an intersection of the possible whereabouts with other (spatial, or spatio-temporal) entities.
%%%In the later section we will have deeper discussion about it.

%%One property of the fusion bead is that the volume is bounded within the single bead as presented
%%%%in figure~\ref{fusing:MAT}. This property guaranteed %%%that fusion model won't incur
%%%false negative in terms of single bead model. In another word, the object's whereabouts are more strictly bounded by fusion model.

%\begin{figure}
%    \centering
%    \includegraphics[width=2.6in]{figure/fusing_mat.png}
%    \caption{Fusion Bead is Bounded by Single Bead}
%    \label{fusing:MAT}
%\end{figure}

\end{comment}
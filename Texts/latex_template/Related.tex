There are two main bodies of research literature that are related to, and were used as
foundation for our work.

The first one consists of results from GIS, MOD and spatio-temporal databases communities, where the problem of
capturing the uncertainty of motion has been studied extensively. Starting with~\cite{hagerstrand},
and more recently~\cite{WinterY10}, the issue of uncertain whereabouts from the perspective
of probabilistic time geography has been tackled by a model of emanating cones-in-time, with a
vertex at the last location sample. The 2D boundary of the possible locations of moving objects with bounded speed was 
formalized by an ellipse in~\cite{pj99}, and its 2D+time version -- beads -- was presented in~\cite{HornsE02}.
Subsequently, \cite{Kuijpers11,Kuijpers10} provided a full formalization of the beads model and also
provided extensions to capture the impact of road networks~\cite{KuijpersMNO10}. Majority of the works
dealing with uncertainty (either in free-space motion or road networks constrained) from MOD and spatio-temporal databases community 
have focused on efficient processing of the popular spatio-temporal queries (range, (k)NN, reverse-NN)
under various models of uncertainty~\cite{GutingS05,ICDE-Tutorial14}.

Unlike majority of the works so far, in this paper we incorporated an additional source of location data -- the 
roadside sensors, and considered the road network which has a width as a parameter, instead of simple edges.

%%Applying space-time prism on road network is a relatively new research area.
%%\cite{Kuijpers:road} investigate the effect of bead model in road network to answer the alibi
%%query. In this work the road network is a graph made of edges and vertices. The experiment 
%%shows that bead model could fulfill the speed limit constraint of road network and calculate
%%the space-range given time constraint. However, in this work the width of road is not taken
%%into consideration. Road network with width is a more complex model that closer to real world
%%situation. On the road network presented in this paper, the movement of vehicle is not
%%simply bounded by graph. Adding width into the model introduce one more degree of freedom.
 
%%Traditional road network model consider the data source as a sequence of space-time sample
%%points. \cite{Kuijpers:2011} extend the bead model to consider the possible acceleration 
%%during the movement. This work further broaden the complex of the model. 
%%However, so far there is no published work that consider merging heterogeneous 
%%data source into a uniform model.


The second body of works originates in the transporation and traffic management communities.
Substantial efforts have been made to tackle the lane-crossing query and 
several works have focused on building novel system to overcome 
the shortcoming of single GPS receivers which yields unstable measurements 
with large uncertainty~\cite{lane_dao,jie_lane}. Attempts have been made to 
acquire location data using commercially available smartphones~\cite{Sekimoto_lane}, however,
nearly 50\% of the data failed to fall within the road network region.
Other efforts include the  use of integrated sensor like gyroscope to fill the
unknown values between two GPS sample updates~\cite{Toledo_lane}.

However, the works did not consider the uncertainty in-between consecutive GPS-based updates and sensor-based
location detections.

Some of the works~\cite{jie_lane,Toledo_lane}, use map matching algorithms to determine 
which lane the vehicle belongs to and, subsequently, try to revise the measurement error using post-processing. 
However, the bead (or, space-time prism) model has not been exploited.


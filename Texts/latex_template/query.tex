Without losing the generality, our discussion focus on a single fused bead.
Notice that the query time interval $[t_{bq},t_{eq}]$ may span over a sequence of beads.
Therefore, one need to check the satisfaction of a given predicate for each bead. This section includes four parts: (1) Basic {\it where\_at} and {\it when\_at} location-in-time (whereabouts) queries (2)
Range queries given a query prism;(3) Lane--crossing query; (4) Pruning techniques that are able to increase
query efficiency and accuracy.
\subsection{Basic Queries}
\begin{figure}
    \centering
    \includegraphics[width=2.8in]{figure/new_bead.png}
    \caption{Whereabouts at Time Instant}
    \label{fusing:bead}
\end{figure}
The possible whereabouts of the object at a given time instant -- i.e., {\it where\_at(oID, t)} query -- for bead obtained by GPS-based samples, is determined via intersecting that bead with a horizontal plane at $t$ (cf. Figure~\ref{Beads-fig}) -- i.e., intersection of two circles centered at $L_1$ and $L_2$ with the radii corresponding to $v_{max}(t - t_1)$ and $v_{max}(t_2 - t)$. 

%%%When fusing the GPS-based location data with the one from the road-side sensors, at each time-instant $t_1 < t \leq t_{s1}$ we have\footnote{We note %%%that the case $t_{s1}$ < $t$ < $t_{2}$ is symmetrical.} the objects whereabouts are located inside (and along the boundary of) the intersection of the %%%horizontal plane at {\it Time = t} and an {\it uncountably infinite union of intersections of pairs of disks} each being bounded by:


%%%\begin{align}
%%%\begin{cases}
%%%t_{i} \leqslant t \leqslant t_{i+1},\\
%%%(x_1 - x_{si})^{2} + (y_1 - y_{si})^{2} \leqslant (t - t_2)^{2}v^{2}_{max},\\
%%%(x_{si} - x_2)^{2} + (y_{si} - y_2)^{2} \leqslant (t_2 - t)^{2}v^{2}_{max},\\
%%%y_{si} = mx_{si} + b, \text{at } t
%%%\end{cases}
%%%\end{align}

Similar to the GPS-based bead, in order to determine the whereabouts at a given time instant $t$ for a fused bead, we need to 
obtain the intersection of {\it FB} with the horizontal plane {\it Time = t}. 
The corresponding illustration of the volume in 2D space + Time, along with the 2D projection, is shown in Figure~\ref{fusing:bead}. The boundary of the 2D projection is obtained as the ``envelope" of the union of two collections of uncountably many pairs of arcs
as discussed in section 4.1.

%%%Each pair of arcs represents the boundaries of the intersections of the corresponding pairs of disks -- one centered at the focus of the GPS-based bead (e.g., $L_1$) and the other centered at a point along the intersection chord (cf. $\overline{L_{11}L_{13}}$ in Figure~\ref{fusing1}) resulting from secant due to the roadside sensor and the arc from the lens of the GPS-based bead. Thus, one of the boundaries is always a circular arc originating at the focal point of the "original" GPS-based bead, centered at focus of the GPS-based bead (say, $L_1$) and with radius $v_{max} (t - t_1)$. The the boundary is actually the boundary of the union of uncountably many disks with radii $v_{max} (t_{s1} - t)$, with centers along the intersection-chord. 

The complementary query, \textit{when\_at(oID, L)} returns the times during which it is possible for the object ${\it oID}$ to be at the location $L(x_L, y_L)$, i.e., a time-interval $[t_L^1, t_L^2]$. The time-interval can be defined as the two intersections between the boundary of the fused bead $FB$ and the vertical line (i.e., ray) emanating from $L$. To calculate the values, we have the following observations:

\begin{enumerate}

\item $t_L^2$ is the latest time that a circle located at the GPS-based focus from the sample at $t_1$ will "reach" $L$ -- hence, it can be obtained as a solution to the equation:

$\overline{L_1L} = v_{max} (t_L^2 - t_1)$

\item $t_L^1$, on the other hand, is the earliest time that any circle with the center on the intersection chord($P_{1}P_{2}$ in figure~\ref{fusing:cross}) and radius $v_{max} (t_s - t_L^1)$  would pass through $L$. 

\end{enumerate}

\begin{comment}
%%%Assuming uniform {\it pdf}s of the possible objects locations within the uncertainty zone defined by the $FB$ model for a given time instant, we now discuss the lane-crossing and range query.  Without loss of generality, we will consider an input consisting of a a single fused bead $FB((x_{i}, y_{i}, t_i), (x_{i+1}, y_{i+1}, t_{i+1}), v_{max}, t_{s}, m, b)$ and a region \textit{R$_{q}$}. 

\end{comment}
\begin{figure}
    \centering
    \begin{subfigure}[b]{0.2\textwidth}
                \includegraphics[width=1.3in]{figure/road1_old.png}
                \caption{Bead Model}
                \label{road1:old}
    \end{subfigure}%
    \begin{subfigure}[b]{0.2\textwidth}
                \includegraphics[width=1.3in]{figure/road1_new.png}
                \caption{Fused Bead Model}
                \label{road1:new}
    \end{subfigure}%
    \caption{MATLAB Visualization}
    \label{road1}
\end{figure}

\begin{comment}
The lane-crossing query  is a minor variation of  {\bf Q1}$^u$  (cf. Section 1):

\noindent {\bf Q}$_{lc}^u$: {\it Retrieve all the vehicles which have  $>$ $\Theta$ ($ 0 < \Theta \leq 1$) probability of crossing
the lane in road segment RS1}.

\begin{figure}
    \centering
    \includegraphics[width=2.6in]{figure/area.png}
    \caption{Evaluating lane-crossing query at $t_{i}$}
    \label{area}
\end{figure}

Let \textit{C$_{t}$} denote the planar region corresponding to the answer of the {\it where\_at(oID, t)} 
and let $f_{L}$ and $f_{e}$ denote the two curves defining the boundary of \textit{C$_{t}$}. Also, let 
$(x_a, y_a)$ and $(x_b, y_b)$ denote the intersection points between $f_{L}$ and $f_{e}$ (i.e., the cusps of 
the boundary of \textit{C$_{t}$}). To calculate the probability that the object $oID$ is crossing the lane at 
time instant $t$, one needs to calculate the area of \textit{C$_{t}$} (\textit{A(C$_{t}$)}) and the area of the 
portion of \textit{C$_{t}$} on "the other side" of the lane. While in some special cases -- e.g., when the GPS-based
location samples are along the line parallel to the lane-separator line (cf. Figure~\ref{area}) and both are axis-parallel -- 
it may be possible to have closed-form formula, we note that, in general one would need to rely on numerical integration.



%%%As illustrated in figure~\ref{area}, 
%%%\textit{C$_{t}$} is bounded by two curves $f_{L}$ and $f_{e}$ -- the corresponding boundaries for the region which 
%%%constitutes the answer to {\it where\_at(oID,t) query.
%%%The area above the x-axis stands for the probability the vehicle 
%%%cross the lane. We observe that these two functions are injective 
%%%functions from Y to X. The intersection points between $f_{L}$ and $f_{e}$
%%%are $y_{a}$ and $y_{b}$. Integral is used in our calculation. 
%%%Because the intersection $y_{a}$
%%%is on the different lane, the integral interval for cross lane area is $[y_{a},0]$.
%%%The area of cross section $C_{t}$ can be calculated by:

%%%$$C_{t}=\int_{y_{b}}^{y_{a}} (f_{e}(x,y)-f_{L}(x,y)) dy$$

%%%The area that represent the probability cross the lane $Area_{t}$ can be calculated by:

%%%$$Area_{t}=\int_{0}^{y_{a}} (f_{e}(x,y)-f_{L}(x,y)) dy$$

%%%A numerical method is essential to implement the above integral since there is no
%%closed-form expression for the envelop function $f_{e}$.
%%We could exam whether the bead cross the lane for a single vehicle using algorithm~\ref{alg:cross}:

%%\begin{algorithm}  
%%  \caption{Vehicle Change Lane}
 %% 	\label{alg:cross}
 %% \begin{algorithmic}[1]
%%	  \For{t in $[t_{i}, t_{i+1}]$}
%%	  	\State  \begin{equation}
%%	  			p = \frac{Area_{t}}{C_{t}}
%%	  			\end{equation}
%%	  	\If {$p$ > $\theta$}
%%	  		\State \textbf{Return} true
%%	  	\EndIf
%%	  \EndFor
%%	  \State \textbf{Return} false 
%%  \end{algorithmic}
%%\end{algorithm}

%%%When it comes to the range query, the methodologies applied for the lane-crossing query would require a minor modification in order to cater
%%to the boundary of the region of interest for a given query (e.g., polygon, circle, etc...). 

\end{comment}

\subsection{Range Query Processing}
Different from {\it Where\_at} queries, for which we are interested in two intersections between boundaries of FB and a vertical line, the range queries ask for the overlapping region between FB and query prism, 
as illustrated in figure~\ref{FB-range}. In this section, for a given range query with parameters: \textit{R}, in the spatial
dimensions; and $[t_{bq}, t_{eq}]$, in the temporal dimension of interest, we denote the set $\{\forall(x, y, t)|(x, y\in R$ and $t \in [t_{bq}, t_{eq}])\}$ for $QP_R$ (query prism). In this work, we focus on query regions that are simple and bounded by convex polygons.

\begin{figure}
    \centering
    \includegraphics[width=2.8in]{figure/FB_range_query.png}
    \caption{Range query for a given prism}
    \label{FB-range}
\end{figure}

Range queries can be classified into different categories~\cite{TrajcevskiMDM10} depending on the containment relationship between beads and query prism
 -- namely, whether the moving object will always be inside \textit{R}, and if it is definitely inside.~\cite{TrajcevskiMDM10} also proposed qualitative approaches to answer existential questions by verifying
intersecting conditions between ellipses and circles. Based on these previous works, we propose a quantitative
range query using probabilistic methodology -- in another word, how much likely moving objects fall into the desired region. 

Given a trajectory, or a sequence of FB $Tr=$ $[FB_1, FB_2, \ldots, FB_n]$, where each
$FB_{i}=$ $((x_i, y_i, t_i), (x_{i+1}, y_{i+1}, t_{i+1}), v_{max}, t_{s}, m, b)$, we
are interested in answering the following range query:\\
\\
\noindent {\bf Q2}$^u$: {\it Given a fused bead $FB(((x_i, y_i, t_i), (x_{i+1}, y_{i+1}, t_{i+1}),$
$ v_{max}, t_{s}, m, b))$, does the moving object have
$>$ $\Theta$ ($ 0 < \Theta \leq 1$) probability of entering $QP_R$}.

\begin{itemize}
\item \textbf{Sometime Inside} $(FB, R, t_{bq}, t_{eq}, \theta, \triangle t)$ 
\end{itemize}

Within the predicate, for a single FB with no refinement, $t_{bq}$ and $t_{eq}$ are the starting and ending query time. In the following section, we will
propose a pruning method to further narrow down the necessary query time interval as $[t_{in}, t_{out}]$.
Therefore, we define a new time interval $[t_{min}, t_{max}]=[t_{in}, t_{out}] \cap [t_{bq}, t_{eq}]$ and assume $[t_{min}, t_{max}] \neq \emptyset$ in the algorithm. $\theta$ is the probability threshold acts as a classification condition when moving
objects enter the query region. Finally, $\triangle t$ is the minimum gap between two time instants we investigate 
in the algorithm. 

The overall probability of entering query region \textit{R} is necessary in order to make a classification. Following
our discussions in section 4.1, the overall probability \textit{Prob} is a triple integral of
\textit{pdfs} on 2D+time space.

$$Prob=\int_{T}\int_{Y}\int_{X} f(x,y,t)$$

The grid based numerical method provides an accurate estimation regarding areas of location whereabouts at a certain
time instant. Because of the uniform distribution of $pdfs$, integral on 2D space is able to be converted as
the ratio between overlapping area and intersection area. Therefore,

$$Prob=\int_{T} \frac{\mbox{Overlapping Area}(t)}{\mbox{Intersection Area}(t)}=\frac{\int_{T}\mbox{Overlapping Area}(t)}{\int_{T}\mbox{Intersection Area}(t)}=\frac{\mbox{Volume of overlapping region}}{\mbox{Overall volume}}$$

The above integral can be concluded as following:

\noindent\textbullet The overall probability of entering query region equals to overlapping volume divided by total volume in 2D+time space.

\begin{algorithm}
  \caption{Sometime Inside $(FB, R, t_{bq}, t_{eq}, \theta, \triangle t)$}
  \label{Range-alg}
  \begin{algorithmic}[1]
      \State Initialize query instant $t_q$;
      \While{$t_q < t_{eq}$}
      	\If {Spatial probability $P_{inside}(t_q) = 1$}
      		\State return: FB enter the region $R$;
      	\Else
      		\State Increment total volume $V_t$ by Intersection Area($t_q$);
      		\State Increment overlapping volume $V_o$ by overlapping Area($t_q$);
      	\EndIf
      \State Increment $t_q$ by $\triangle t$;
      \EndWhile
      \State Overall probability $Prob = \frac{V_o}{V_t}$;
      \If {$Prob \geq \theta$}
      	\State return: FB enter the region $R$;
      \Else
      	\State return: FB does not enter the region $R$;
      \EndIf
  \end{algorithmic}
\end{algorithm}

In algorithm~\ref{Range-alg}, FB is sliced into flat cylinders, whose volume can be estimated as bottom area multiplied
by height. Therefore, 

$$Prob=\frac{\sum \mbox{overlapping Area} \times h}{\sum \mbox{Intersection Area} \times h}$$

Line 6 and 7 in algorithm~\ref{Range-alg} implement aggregations for overlapping and intersection areas.

In additional, we check a special case when $P_{inside}(t_q)=\int_{Y}\int_{X} f(x,y,t)=1$. In this
situation, the moving object completely cross the boundary and enter the query region.
As a result, we stop the running loop and return true --- for the purpose of eliminating false negatives
that potentially might happen when
one GPS point barely cross the boundary. There are two advantages of algorithm~\ref{Range-alg}: (1) no additional
false negatives are introduced; (2) it takes advantages of FB model, and the number of false positives are reduced.

%need more here (discussion about false negative and checking techniques)

\subsection{Lane-crossing Query Processing}
Lane-crossing query is a very special application of range query, where the query prism is degenerated from
a polygon into a half plane. As shown in figure~\ref{fusing1-1}, the query region \textit{R} is the half-plane
above the dashed blue line.

In the introduction section, we have stated lane-crossing query {\bf Q1}$^u$,
which is important in applications related to efficient traffic management. We reiterate {\bf Q1}$^u$
and convert it into a form that is applicable to single FB.\\
\\
\noindent {\bf Q1'}$^u$: {\it Given a fused bead $FB(((x_i, y_i, t_i), (x_{i+1}, y_{i+1}, t_{i+1}),$
$ v_{max}, t_{s}, m, b))$, does the moving object have
$>$ $\Theta$ ($ 0 < \Theta \leq 1$) probability of crossing the lane and entering half-plane $R$}.\\
\\
Algorithm~\ref{Range-alg} can be Applied to query {\bf Q1'}$^u$. The overall result to {\bf Q1}$^u$ is
the aggregation of each individual return from {\bf Q1'}$^u$.

\subsection{Pruning Techniques}
There are usually three stages for spatial-temporal query processing~\cite{GutingS05}:\\

\noindent(1) \textit{Filtering}, where index is used to eliminate those data items that are
guaranteed not to satisfy the query~\cite{TaoXC07,TrajcevskiMDM10};\\
\noindent(2) \textit{Pruning}, where some properties might be used to further filter out a portion
of the data set without introducing any false negatives;\\
\noindent(3) \textit{Refinement}, where certain algorithms are used to eliminate false positives that were not filtered out during the previous stage(s).\\

In this work, we do not address the problem of indexing. In the sequel, we focus on the last two stages of query processing. Algorithm~\ref{Range-alg} for range query, as an example, belongs to the refinement stage. In order to
boost the overall execution time, certain pruning techniques are essential.


\noindent\textit{A. Definitely Outside -- Individual Fused Bead Bounds (IBb)}

First proposed by~\cite{TrajcevskiMDM10}, this pruning strategy is originally designed for GPS-based bead.
It approximates each GPS-based bead with its minimum bounding vertical cylinder. According to the Lemma~\ref{bead_bond}, FB is bounded by GPS-based bead, which inspires us to extend the application of IBb to FB as well.
In effect, the ellipse -- which is the projection of a bead, formed by two GPS points from FB, becomes a circle centered in the center of the respective ellipse, as shown in figure~\ref{IFBb}. The radius of the approximation-disk $Ad_i$ is: $r(Ad_i) = 1/2(v_{max}^i)(t_{i+1} - t_{i})$.

\begin{figure}
    \centering
    \includegraphics[width=2.8in]{figure/IFBb.png}
    \caption{Cylinder-based pruning approximation}
    \label{IFBb}
\end{figure}

\noindent\textit{B. Definitely Inside -- GPS points pre-screening}

This pruning technique is specially designed for lane-crossing query, where the predicate determines if the FB is possible to cross the lane. We have the following observation, namely, if two GPS points are located on two sides
of the central line, there must be at least one time, that the moving object crosses the road, in which case we are
able to prune the FB and direct return true. This pre-screening, together with algorithm~\ref{Range-alg}, further
guarantees the fact that FB does not produce false negative, since all scenarios when moving objects cross the lane according to linear interpolation will be classified as true.

\noindent\textit{C. Sometime Inside -- fine grained dead-space removal}

We are interested in finding
the time instant when the uncertain trajectory enter/exit the query region $R$, to be
more specific, in finding critical points. By doing so, we eliminate some redundant time before and after query processing.

The general case for time $t \in [t_{i}, t_{i+1}]$
being a critical point occurs when the intersection of the uncertain region at $t$ with a query rectangle is a single point. In the time interval $[t_{i}, t_s]$, the single-point-intersection between disk centered at the first GPS point and query region stands for the entering moment. Similarly, in the time interval $[t_s, t_{i+1}]$, the single-point-intersection represents exiting moment. Since the query region is represented as polygon in the $(X,Y)$ plane, each edge of the polygon is defined as a segment of 2D line $y = ax+b$. 

The entry boundary of FB is:

$$(x - x_{i})^{2} + (y - y_{i})^{2} = (t - t_{i})^{2}v^{2}_{max}$$

Substituting for y for the equation of the line, we have:

$$(x - x_{i})^{2} + (ax+b - y_{i})^{2} = (t - t_{i})^{2}v^{2}_{max}$$

This yields an equation in x and t:

$$A*x^2+x*(B+C*t)+D*t^2+E=0$$

Where $A,B,C,D,E$ are constant. Solving for x, as a function of t, we have:

$$x_{1,2}=\frac{-(B+C*t) \pm \sqrt{(B+C*t)^2-4*A*(D*t^2+E)}}{2*A}$$

To be noted that, we need to check solutions for $x$ against boundaries of the respect
edge of the query region. To find the time for critical point, we set the discriminant to
be zero:

$$\sqrt{(B+C*t)^2-4*A*(D*t^2+E)}=0$$

The real root $t_{in}$ is the time instant when the uncertain trajectory start to enter the query prism.

For time value $t \in [t_s, t_{i+1}]$, we can use the similar method to find the exiting time $t_{out}$. Our new
query time interval in algorithm~\ref{Range-alg} is:
$$[t_{min},t_{max}] = [t_{in},t_{out}] \cap [t_{i},t_{i+1}]$$
\begin{comment}
We now proceed with elaborating some basic calculations regarding the boundary of the possible locations of a given object at a specific time 
instant under the FB model, as well as the time-interval during which an object can be at a particular location. Subsequently, we also discuss
the methodology for detecting whether the possible locations of a moving object are part of a given (spatial) range.


%%There are several fundamental questions we need to solve before processing different types of queries.
%%%What is the location whereabouts at given time instance in FB model and how to calculate it? How do we define the a moving object belongs to 
the %%%query range? In this section we will have multiple theatrical preparations for FB query.



\subsection{Fused Bead -- Shape at Time Instant} \label{whereabouts}

Recall that the FB model is based on the original bead obtained via
GPS-based locations $L_1$ and $L_2$ (foci of a 2D ellipse $El_1$) and a road-side sensor providing possible
locations along a line-segment perpendicular to a given road at a time instant $t_{s}$ (cf. Figure~\ref{fusing1-1}).

%%%The composition of fused bead based on constraint (1) is shown in figure~\ref{fusing1-1}. The original
%%%GPS-based location $L_1$ and $L_2$ yield a 2D projection which is a ellipse having them
%%%as foci, noted as $El_1$. The road-side sensor provide
%%%location information at $t_{s}$ along the
%%%line that is perpendicular to the road with x-axis $X_{s1}$. $S_1$ intersect with $El_1$
%%%with uncountably many points. Each point, such as $L_{11}$, paired with original foci of $El_1$ could
%%%be regarded as a new generator of two more space-time prism that one start at $L_1$
%%%and one terminate at $L_2$. The Fused bead is the union of uncountably many small bead
%%%with one anchor point at foci of original ellipse and one moving point along $S_1$.

%%%However, as we try to answer the query for the
When it comes to location whereabouts at certain time instant $t_{s1}$, the regular bead model has a boundary defined
%%%there is another angle to tackle the problem, namely,
%%%instead of considering the union of uncountably many bead, the projection of fused bead is
%%%the union of location whereabouts at each time instance in $[t_i, t_{i+1}]$. Recall that,
%%%at any given time instance $t_{s1}$, the location whereabouts of GPS-based bead is
by a lens $L_e(t_{s1})$ which obtained as the intersection of the circles with radii
$v_{max}(t_{s1}-t_{1})$
and $v_{max}(t_{2}-t_{s1})$, centered at $L_1$ and $L_2$ respectively (light blue shaded area in Figure~\ref{fusing:cross}). If it happens that
at that same time instant the object has been detected by a roadside sensor -- then the object must be somewhere along the ray emanating from 
that sensors location and perpendicular to the road segment. However, because of the uncertainty boundary from the GPS-based location data, only 
the points along that ray which are inside the lens $L_e$ are valid possible-locations -- illustrated by the segment $P_1P_2$ in 
Figure~\ref{fusing:cross}.
%%%its location
%%%The ray emanating from the road-side sensor intersect lens $L_e$ at $P_1$ and $P_2$ in Figure~\ref{fusing:cross}, and the line segment $P_1 
%%%P_2 %%%is the valid
%%%location whereabouts at $t_{s1}$.


\begin{figure}
    \centering
    \includegraphics[width=2.8in]{figure/crosssection.png}
    \caption{Cross Section of Fused Bead}
    \label{fusing:cross}
\end{figure}

Let $\varepsilon \in [0,1]$ denote a real variable. Any point $P(t_{s1},\varepsilon) \in \overline{P_1P_2}$ which is a possible location of the 
object at $t_{s1}$ has coordinates $x_{P(\varepsilon} = \varepsilon x_{P1} + (1 - \varepsilon) x_{P2}$ and $y_{P(\varepsilon} = \varepsilon 
y_{P1} + (1 - \varepsilon) y_{P2}$

With this in mind, given a time instant $t_i \in$  $[t_1, t_{s1}]$, the possible locations of the moving object at $t_i$ are bounded by the 
uncountable union of  intersections between:

\begin{enumerate}

\item The disk centered at $L_1$ and with radius $v_{max}(t_{i}-t_{1})$.

\item An infinite collection of disks, each centered at a point $P(t_{s1},\varepsilon)$ along $P_1P_2$ and each with radius 
    $v_{max}(t_{s}-t_{i})$

\end{enumerate}

%%%two collections of uncountably many pairs of arcs. Each pairs of arcs represents the boundaries of the intersections of the corresponding 
pairs o%f disks---one collection of arcs
%%centered at one foci of the GPS-based bead (e.g., $L_1$ in figure~\ref{fusing:cross})
%%and another one centered along line segment $P_1 P_2$. As the result, one of the boundaries is
%%a circular arc with radii $v_{max}(t_i-t_1)$, centered at the focus of the GPS-based bead.

In Figure~\ref{fusing:cross}, the circles $C_1, C_2$ and $C_3$ are examples of the boundaries of the objects whereabouts at different 
time-values ($t_i$) due to the GPS-sample at location $L_1$. For a fixed value of $t_i$ Figure~\ref{fusing:cross} also shows the boundary 
defined by the "envelope" of the union of
the uncountably many disks centered along $P_1 P_2$ -- essentially, the sum of the line segment $P_1 P_2$ and a disk with radius 
$v_{max}(t_s-t_i)$.
%%% and centered on the intersection-chord,as shown
%%%in figure~\ref{fusing:cross}.
%%%We use $C_{ap}(t_i)$ to denote this Capsule-like shape at a given time instant $t_i$.
%%%This union of uncountably many disks results in a capsule shape area, being remarked as $C_{ap}$, made of two half circles and a rectangle.

Depending on the time value and $\varepsilon$, there are five basic kinds of time-intervals during which the shapes the unions determining the 
object's whereabouts have distinct properties. We use the phrase {\it significant times} to denote the boundaries of those time-intervals. 
%%between the disks composing $C_{ap}(t_i)$ and the %%disk centered at $L_1$:

\begin{enumerate}

\item $t \in [t_1, t_i^{l1})$ ({\it from disks until the first lens}): During this interval, the possible locations are inside a disk centered 
    at $L_1$ -- this is the case when $t_i$ is very close to $t_1$, meaning: regardless of the value of $\varepsilon$, each disk with radius 
    $v_{max}(t_{s}-t_{i})$ centered at any point along $P_1, P_2$, fully covers the disk centered at $L_1$ with radius $v_{max}(t_{i}-t_{1})$. 
    Let $P_c$ denote the point along $P_1 P_2$ which is geometrically closest to $L_1$. Clearly, this point will be the one with the earliest 
    change of this kind of containment with the disk -- at some time instant $t_i^{l1}$, the intersection\footnote{For clarity, we present the 
    details of calculating $t_{i}^{l1}$ and other significant times in the Appendix.} will switch from a full-disk centered at $L_1$ into a 
    lens defined by the intersection of the two disks: one centered at $L_1$ and one centered at $P_c$.

\item $t \in [t_i^{l1}, t_i^{l_A})$ ({\it from a single lens, until "lenses\_All"}): During this time interval, depending on the values of 
    $\varepsilon$, some of the disks centered along $P_1 P_2$ (each with radius $v_{max}(t_{s1}-t_{i})$) are still fully covering the disk 
    centered at $L_1$ with radius $v_{max}(t_{i}-t_{1})$. These are the ones whose centers are closer to $P_1$ (i.e., $P(t_{s1},\varepsilon)$ 
    with $\varepsilon$ closer to $0$).

\item $t \in [t_i^{l_A}, t_i^{d1})$ ({ \it from lenses\_All, until the first (full) disk appears}): This is the time-period during which each 
    possible foci along $P_1 P_2$ is a center of a disk with which yields a lens-shaped intersection with the disc centered at $L_1$. At the 
    expiration of this time interval, the disk centered at $P_c$ and with radius $v_{max}(t_{s}-t_{i})$ is about to be fully covered by the 
    disk centered at $L_1$ and with radius: $v_{max}(t_{i}-t_{1})$

\item $t \in [t_i^{d1}, t_i^{d_A})$ ({ \it from a single disk appearance, until disks\_All}): similarly to the 2nd case above, during this 
    time interval some of the disks centered along $P_1 P_2$ have a lens-shaped intersection with the disk centered at $L_1$, while some are 
    fully contained inside of it.

\item $t \in [t_i^{d_A}, t_{s1})$ ({ \it disks\_All}): The last distinct time-interval for the part of the FB between the first GPS-based foci 
    and the roadside sensor is similar to case "1." above, in the sense that every disk with radius $v_{max}(t_{s1}-t_{i})$, regardless of 
    where its center is located along $P_1 P_2$, is fully contained inside the disk centered at $L_1$ and with radius  
    $v_{max}(t_{i}-t_{1})$.


\end{enumerate}


We note that for time-values $t \in [t_{s1}, t_2]$, the cases are analogous (and in reverse order) from the ones specified above, in the sense 
that there are four significant time instants defining five distinct intervals.

%%We note that, while each of the intersection points may be calculated by detecting the %%%intersections of two circles, or a circle and a line 
%%segment, calculating the value of %%the intersection of the possible whereabouts with another region (e.g., for query %%processing) may require 
%%numerical calculations, which we address next.

%%The capsule $C_{ap}$ could be described by two circle functions and two line segment functions when we analyze the intersection scenario. 
%%However, formalize this intersection is unrealistic considering its variation. As the result, in our implementation, a special numerical 
%%method %%is pursued to build the intersection area at given time instance. %we build
%the fused bead model in space-time coordinate, and the cross section can be obtained by
%intersecting fused bead with a horizontal plane as shown in figure~\ref{fusing:3d}.


%%%\subsection{NEED NEW TITLE HERE...}



%FB accompanies uncertainty, and in nature we are unable to conclude an accurate location point at given time
%instance, but a possible whereabouts region. How to define the case that the query object move into the query
%range become critical. Even though we have the location whereabouts as discussed in section 4.1 and an overlap between location whereabouts and 
%query range exist, we are still unable to label the predicate whether
%the object is inside or not. Therefore, in this section, we pursue a probabilistic method and use probability
%to describe the likelihood the moving object belongs to query range at certain time instance.

%%%We proceed with introducing a few properties of fused bead that will assisting the calculation.
%%%Recall that for a given fused bead $FB ( (x_i,y_i,t_i), (x_{i+1},y_{i+1},t_{i+1}), %%%v_{max},t_{s},m,b)$, the system of equations used to 
%%determine the (boundary of the) possible whereabouts is:
%%%he ray emanating from the road-side sensor intersects lens $L_e$ at $P_1$ and $P_2$. The simultaneous equations contain lens and line are:

%%\begin{align}
%%\begin{cases}
%%(x - x_{1})^{2} + (y - y_{1})^{2} = (t - t_{1})^{2}v^{2}_{max},\\
%%(x - x_{2})^{2} + (y - y_{2})^{2} = (t_{2} - t)^{2}v^{2}_{max},\\
%%y=ax+b.
%%\label{lens}
%%\end{cases}
%%\end{align}

%%\begin{figure}
%%    \centering
%%    \includegraphics[width=2.8in]{figure/range.png}
%%    \caption{Intersection of Uncertain Region and Query Range}
%%    \label{fusing:range}
%%\end{figure}

%%%We can obtain the points $P_1$ and $P_2$ by simply finding the intersections of the $y %%= ax + b$ with each of the circles, call them 
%%$I_{l1}, I_{l2}, I_{u1}$ and $I_{u2}$, %%%and selecting the two of them (e.g., $I_{u1}$ and $I_{l2}$).
%%%corresponding to line $S_1$ intersects two circles. The line segment
%%%$P_1 P_2$ is the intersection of two line segments $I_{l1} I_{l2}$ and $I_{u1} %%I_{u2}$.
%%%As a consequence,
%%%we could conclude that $P_1(x_{p1}, y_{p1})$ equals $I_{u1}$ and $P_2(x_{p2}, y_{p2})$ equals $I_{l2}$. The length of $P_1 P_2$ is denoted as 
%%$L$.
%%One important observation about the fused bead model is that, there are two location-fixed time at $t_1$ and $t_2$ and a location-semi-fixed 
%%time %%at $t_s$.
%%Although there are two portions of the FB (from $t_1$ to $t_{s1}$, and from $t_{s1}$ to %%%$t_2$) the analysis is, in some sense, symmetric -- 
%5so, in the sequel we focus only on %%deriving the specifics for the interval $[ t_1, t_{s1}]$.
%%%The location whereabouts
%%%start from $t_1$, converge to a line segment at $t_s$, and finally stop at $t_2$. When %%%processing a single fused bead, we separate it into 
%%two %%%half-beads. Although these %%%two
%%%half-beads will not always be symmetric (only in the case the $S_1$ is the minor axis %%of
%%%ellipse), they are composed in the identical manner, and the query process is %%similar.

%%After finding the line segment $P_1 P_2$, let $d_{i1} = %%\sqrt{(x_{p1}-x_{1})^2+(y_{p1}-y_{1})^2}$ be the distance between the start location 
%%at $t_1$ and end location at $t_s$.
%%Besides, let $t_{svl_{1}} = (t_1+t_s)/2-d_{i1}/2v_{max}$ and $t_{svl_{2}} = %%(t_1+t_s)/2+d_{i1}/2v_{max}$. The uncertain region $UR_t$ for a 
%%given time instance $t$ %%between
%%$[t_1, t_s]$ could be divided into three phases:

%%(1) Single disk. For every $t\in[t_1, t_{svl_{1}}]$, the uncertain region at t is a %%disk
%%with radii $r(t)=v_{max}(t-t_1)$, centered at $(x_1, y_1)$.

%%(2) Intersection between circle and capsule. For every $t\in[t_{svl_{1}}, %%t_{svl_{2}}]$,
%%the uncertain region is the intersection between circle and capsule. The circle is %%centered
%%at $(x_1, y_1)$ with radii $r(t)=v_{max}(t-t_1)$. The capsule, as we discussed in the %%previous
%%section, is the union of uncountably many circles centered along line segment $P_1 P_2$ %%with
%%radii $r(t)=v_{max}(t_s-t)$.

%%(3) Capsule. The uncertain region for every $t\in[t_{svl_{2}}, t_s]$ is a capsule %%formed
%%by union of uncountably many circles with radii $r(t)=v_{max}(t_s-t)$, centered along %%$P_1 P_2$.

%%Another half-bead corresponding to the time interval $[t_s, t_2]$ can be divided into %%similar
%%three parts as well.

%%Assuming the moving object has the same chance of being located in every points in the %%uncertain area. In another word, the probability 
%%distribution on uncertain region %%follows
%%the uniform distribution. The probability density function $f(x,y,t)$ is specified as %%following:

Let $D_1(t)$ denote the disk centered at $L_1$ and with radius $v_{max}(t-t_1)$. Also,
let $D_P(t,\varepsilon)$ denote the disk centered at the point $P(t_{s1},\varepsilon)$ with radius $v_{max}(t_{s1}-t)$.
Assuming a uniform distribution in each time-interval between two consecutive significant times\footnote{Throughout this work, we assume 
independence between location-values in successive location samples (cf.~\cite{ICDE-Tutorial14,EmrichKMRZ12}.}, we obtain that the corresponding 
{\it pdf}s (probability density functions) are:

\begin{enumerate}

\item $t \in [t_1, t_i^{l1})$:


$$f(x,y,t)=\begin{cases} \frac{1}{\pi(v_{max}(t-t_1))^2} & \mbox{if} (x,y)\in D_1 (t)\\0& \mbox{otherwise}\end{cases}$$

\item $t \in [t_i^{l1}, t_i^{l_A})$

$$f(x,y,t) = \frac{1}{\pi(v_{max}(t-t_1))^2 + A(\cup_{\varepsilon > \delta_1(t)}(D_1(t) \cap D_P(t,\varepsilon)))}$$

where $\delta_1(t)$ is the smallest value of $\varepsilon$ at a given $t$ for which $D_P(t,\varepsilon) \not\subseteq D_1(t)$. A stands for operations to calculate areas.

\item $t \in [t_i^{l_A}, t_i^{d1})$

$$f(x,y,t) = \frac{1}{A(\cup_{\varepsilon }(D_1(t) \cap D_P(t,\varepsilon)))}$$


\item $t \in [t_i^{d1}, t_i^{d_A})$

$$f(x,y,t) = \frac{1}{\pi(v_{max}(t_{s1}-t))^2 + A(\cup_{\varepsilon > \delta_2(t)}(D_1(t) \cap D_P(t,\varepsilon)))}$$

where $\delta_2(t)$ is the smallest value of $\varepsilon$ at the given $t$ for which $D_1(t) \not\subseteq D_P(t,\varepsilon)$.

\item $t \in [t_i^{d_A}, t_{s1})$

$$f(x,y,t)=\begin{cases} \frac{1}{\pi(v_{max}(t_{s1}-t))^2 + \overline{P_{1}P_{2}} \cdot (v_{max}(t_{s1}-t)) } & \mbox{if}(\forall \varepsilon) D_P(t,\varepsilon) \subseteq D_1(t) \\ 0& 
\mbox{otherwise}\end{cases}$$

\end{enumerate}

When calculating the probability that a given moving object whose motion is modelled as an FB is inside a given spatial range at a given time 
instant, we need the area of the intersection and, generically speaking:

%%We get a similar formula for $t\in[t_{svl_{2}}, t_s]$:
%%$$f(x,y,t)=\begin{cases} \frac{1}{\pi(v_{max}(t_s-t))^2+v_{max}L(t_s-t)} & \mbox{if} (x,y)\in D_2 (t)\\0 & \mbox{otherwise}\end{cases}$$

%%(2) For $t\in[t_{svl_{1}}, t_{svl_{2}}]$, we observe that the probability density function
%%$f(x,y,t)$ can be regarded as joint distribution of two random variables $X$ and $Y$, where
%%$X$ and $Y$ both have uniform distribution:
%%$$f_X(x,y,t)=\begin{cases} \frac{1}{\pi(v_{max}(t-t_1))^2} & \mbox{if} (x,y)\in D_1 (t)\\0& \mbox{otherwise}\end{cases}$$

%%$$f_Y(x,y,t)=\begin{cases} \frac{1}{\pi(v_{max}(t_s-t))^2+v_{max}L(t_s-t)} & \mbox{if} (x,y)\in D_2 (t)\\0 & \mbox{otherwise}\end{cases}$$
%%The joint density function can be concluded as:
%%\begin{widetext}
%%\begin{equation}	
%%f(x,y,t)=\begin{cases} w_1*\frac{1}{\pi(v_{max}(t-t_1))^2} & \mbox{if} (x,y)\in D_1 (t)\cap \overline{D_2}\\
%%w_2*\frac{1}{\pi(v_{max}(t_s-t))^2+v_{max}L(t_s-t)} & \mbox{if} (x,y)\in D_2 (t)\cap \overline{D_1}\\
%%w_1*\frac{1}{\pi(v_{max}(t-t_1))^2} + w_2*\frac{1}{\pi(v_{max}(t_s-t))^2+v_{max}L(t_s-t)} & \mbox{if} (x,y)\in D_1 (t)\cap D_2
%%\end{cases}
%%\end{equation}
%%\end{widetext}
%%Where $w_1 = \frac{\pi(v_{max}(t-t_1))^2}{\pi(v_{max}(t-t_1))^2+\pi(v_{max}(t_s-t))^2+v_{max}L(t_s-t)}$ and $w_2=1-w_1$.
%%
%%Therefore, we are able to calculate the probability value.

%%For $t\in[t_1, t_{svl_{1}}]$
\begin{comment}
\begin{equation} \label{integral}
P_{inside}(B, R, t) =\int_{UR_t \cap R}f(x,y,t)dxdy \\
= \frac{Area\mbox{ }of\mbox{ }{\textit{UR_t}}\mbox{ }that\mbox{ }overlaps\mbox{ }R}{Area\mbox{ }of\mbox{ }\{textit{UR_t}}}
\end{equation}
\end{comment}

%%For $t\in[t_{svl_{1}}, t_{svl_{2}}]$, the equation~\ref{integral} still holds to calculate the probability.

However, given the complexity of the boundary of the objects whereabouts, the calculation of overlapping area necessitates relying on numeric 
integration methods and getting approximate answers.


\subsection{Numerical Method for Complex Area Calculation}




Numerical methods provide a way to solve problems quickly and easily compared to analytic methods, especially in some cases when the analytic 
methods are too complex or
unachievable, in our case, estimating the overlap area using numerical methods is easy. We determine the type of numerical methods according to the 
aim of tasks.
If we try to obtain the intersection of two curves, the Newton-Raphson Method
is the most widely used one. If the area under a curve is desired, Trapezoid Rule, Gaussian
Quadrature Method or Monte Carlo Integration are useful tools.
\begin{figure}
    \centering
    \includegraphics[width=2.8in]{figure/old_area.png}
    \caption{Area calculation in GPS-based bead}
    \label{old_area}
\end{figure}
In GPS-based bead, the location whereabouts given time instance is a relatively simple
geometry---either a circular disk or a lens formed by intersection of two circles as
shown in figure~\ref{old_area}. Intersection points $P_1$ and $P_4$ between two curves
$f_1(x, y)$ and $f_2(x, y)$ can be obtained by collecting two implicit functions and applying
Newton-Raphson Method.

\begin{align}
\begin{cases}
f_1(x, y)=0\\
f_2(x, y)=0
\label{equset1}
\end{cases}
\end{align}

Similarly, Intersection points $P_2$ and $P_3$ between line segment $f_3(x, y)$
of query region $R$ and two circles can be obtained from equation sets~\ref{equset2}~\ref{equset3}.

\begin{align}
\begin{cases}
f_1(x, y)=0\\
f_3(x, y)=0
\label{equset2}
\end{cases}
\end{align}

\begin{align}
\begin{cases}
f_2(x, y)=0\\
f_3(x, y)=0
\label{equset3}
\end{cases}
\end{align}

The total area $UR$ can be regarded as two separate segments, each of them will be attained
from circle sector minus a triangle. The overlap area between $UR$ and query region is the
union of two circle segments and a triangle. In a word, the approach to calculate the cross section area of GPS-based bead is the combination of 
numerical method, used to find intersection points, and basic geometry.

However, after incorporating road-side sensor into our model and transform from GPS-based bead to
fused bead, our traditional slicing and pasting method does not work. As we discussed in
section 4.1,
the intersection points between circle and capsule continuously changing following four cases: (a) There are
two intersection points on the left line segment; (b) There is an intersection point on the left line segment and a point on half-circle; (c) 
There is an intersection point on the right line segment and
one or two points on half-circle; (d) There is an intersection point on the right line segment. In these scenarios, the components of cross 
section area $UR$ is uncertain, and slicing it into basic geometry is not trivial.
\begin{figure}
    \centering
    \includegraphics[width=2.8in]{figure/grid.png}
    \caption{Grid Based numerical approximation}
    \label{grid}
\end{figure}
To overcome the uncertain geometrical composition challenge, we propose grid based numerical approximation. We have the following observation, 
the capsule shape is
the minkowski sum of uncountably many circles with center moving along the ray emanating from the roadside senor. Based on this observation, our 
philosophy behind the grid approach is to
allow cross section area grow as it supposed to be. In another word, instead of analyzing its geometry
component at given time instance, we calculating the overall location whereabouts by
aggregrating areas of each sub-bead. Figure~\ref{grid} explain this mechanism, the yellow circle is moving along
the sensor line, starting from $p_1$ and ending at $p_2$. We set a moving step $\Delta d$
as the minimum moving step so that the number of circles being involved in calculation is reduced from infinitely uncountable many to finite 
countable. When the circle moves to $c_1$, it intersects with fixed circle $c$ centered at one foci of original ellipse. Their intersection 
$c\cap c_1$ is captured by grid. In the next step, it moves to $c_2$ and we have their intersection $c\cap c_2$. The sum is obtained 
by:

$$(c\cap c_1)\cup (c\cap c_2)\cup ...\cup (c\cap c_n)$$
\noindent
One example of the above union operation is grid $g$ in figure~\ref{grid}. Grid $g$ is not covered by $c\cap c_1$ but $c\cap c_2$. As the 
circle moving forward, more and more grids
belong to cross section area are covered, and finally we achieve a numerical approximation
of cross section. The area of grid shape can be calculated through simple counting or more advanced optimization technique to get a better 
approximation. For example, the oblique boundary curve approximation is a boundary approximation technique with high accuracy.

To calculate the area of intersection between $UR$ and query region $R$, we simply perform
$$R\cap[(c\cap c_1)\cup (c\cap c_2)\cup ...\cup (c\cap c_n)]$$
Since it is relatively easy to determine if a grid is in the region $R$ or not, we convert
the difficult area calculation into a trivial question with set operation and counting.

\begin{figure}
    \centering
    \includegraphics[width=2.8in]{figure/grid_ill.png}
    \caption{Shape Density and Its Boundary}
    \label{grid-shape}
\end{figure}

Now, we shall proceed to discuss how to calculate the area approximation. In figure~\ref{grid-shape}, circular shape $m$ is overlaid on grid 
$M$. Let the
area of grid cell be $A_m$ and let the area belongs to shape $m$ be $\Omega$. The density
function can be calculated by:

\begin{align}
f(x,y)=
\begin{cases}
1 & \mbox{if } A_m \cap \Omega = A_m\\
0 & \mbox{if } A_m \cap \Omega =\emptyset\\
a_m/A_m & \mbox{if } A_m \in \Gamma
\label{shape}
\end{cases}
\end{align}

\noindent
Although the approximation looks straight forward, the challenge exists for those elements reside
on the boundary $\Gamma$. As we investigate those grid cells on the boundary of shape, some parts of the cell $a_m$ belong to shape, and the 
density of the boundary cell can be
described by fraction $a_m/A_m$. The shape area is the aggregation of all grid cell multiplied by their density function.

There are two directions to improve the shape approximation accuracy. On one hand, we could
increase the level of granularity to achieve a fine grained model; on the other hand,
a better boundary estimation technique is desired since the approximation errors come from
boundary cells.
\end{comment}